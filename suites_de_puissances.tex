\documentclass{amsart}
\usepackage[french]{babel}
\usepackage{tabularx}

\newtheorem{theorem}{Théorème}[section]
\newtheorem{lemma}[theorem]{Lemme}

\theoremstyle{definition}
\newtheorem{definition}[theorem]{Définition}
\newtheorem{example}[theorem]{Exemple}
\newtheorem{xca}[theorem]{Exercice}

\theoremstyle{remark}
\newtheorem{remark}[theorem]{Remarque}

\numberwithin{equation}{section}

%    Absolute value notation
\newcommand{\abs}[1]{\lvert#1\rvert}

%    Blank box placeholder for figures (to avoid requiring any
%    particular graphics capabilities for printing this document).
\newcommand{\blankbox}[2]{%
  \parbox{\columnwidth}{\centering
%    Set fboxsep to 0 so that the actual size of the box will match the
%    given measurements more closely.
    \setlength{\fboxsep}{0pt}%
    \fbox{\raisebox{0pt}[#2]{\hspace{#1}}}%
  }%
}

\begin{document}

\title{Suites de puissances}

\author{VERTUEUX}

\begin{abstract}
    Cet article est une énonciation et démonstration de propriétés sur des suites ayant pour forme $u_{n+1}=u_n^r$ avec $u_0=k$, $k \in \mathbb{R}$, $r \in \mathbb{R^*}$.
\end{abstract}

\maketitle

\begin{definition}
    On appelle une suite de puissances une suite $u_n:\mathbb{N} \rightarrow \mathbb{R}$, $\forall n \in \mathbb{N}$, $n \mapsto u(n)$ avec $k \in \mathbb{R}$, $r \in \mathbb{R^*}$, ayant pour forme: 

    \[\begin{cases} 
        u_{n+1}=u_n^r \\
        u_0=k
    \end{cases}\]
\end{definition}

\begin{definition}
    Toute suite $u_n$ étant une suite de puissances, $r$ est appelé raison de la suite.
\end{definition}

\begin{theorem}
    Toute suite $u_n$, une suite de puissances de formule de récurrence $u_{n+1}=u_n^r$ avec $u_0=k$ peut s'écrire de la forme $u_n=k^{r^n}$, avec $k \in \mathbb{R}$, $r \in \mathbb{R^*}$.
\end{theorem}

\begin{proof}
    À l'initialisation de la suite $u_n$, nous avons $u_0 = k$, et $u_1 = k^r$.

    Par récurrence, nous savons que $u_2 = (k^r)^r$ et que $\forall n \in \mathbb{N}$:
    
    \[u_n = k^{\prod_{i=1}^{n}r} \iff u_n = k^{r^n}\]

    Dans le cas où $r = 0$, on remarque que: 

    \[\begin{cases} 
        u_{n+1}=1 \\
        u_0=k
    \end{cases}\]

    Et d'après la formule précédente, $u_n = k^{0^n} \implies u_0$ est indéfini.

    Or, $u_0 = k$, $k \in \mathbb{N}$ d'après la formule de récurrence. Par conséquent, la suite ne peut pas être écrite avec une formule explicite lorsque $r=0$.

\end{proof} 

\begin{theorem}
    Toute suite $u_n$, une suite de puissances de formule de récurrence $u_{n+1}=u_n^r$ avec $u_0=k$, et un rang $u_p$, $p \in \mathbb{N}$, peut s'exprimer par: $u_n = u_p^{r^{n-p}}$.
\end{theorem}

\begin{proof}
    Considérons trois cas: $p<n$, $p>n$, $p=n$.

    Lorsque $p<n$, on remarque qu'afin d'atteindre $p$, il suffit de multiplier par la raison $r$ selon la distance séparant $p$ et $n$, soit $n-p$:
    
    \[u_n = u_p^{\prod_{i=1}^{n-p}r} \iff u_n = u_p^{r^{n-p}}\]
    
    Lorsque $p>n \iff n<p$, de manière analogue à $p<n$, on sait que:

    \[u_p = u_n^{\prod_{i=1}^{p-n}r} \iff u_p = u_n^{r^{p-n}} \iff u_n = u_p^{r^{n-p}}\]


    Lorsque $p=n$, les termes sont de même rang $\implies u_n=u_p$. En considérant le Théorème énoncé aussitôt:

    \[u_n = u_p^{r^{n-p}} \iff u_n=up^{r^0} \iff u_n=u_p, \forall r \ne 0\]

    Nous retombons bien sur l'égalité avec $u_n=u_p$.
    
    À noter de même que dans le cas où $p=0$, on a bien: $u_n=u_0^{r^n}=k^{r^n}$.

    Connaissant $u_p$ et $u_n$, on peut donc retrouver la raison $r$:

    \[u_n = u_p^{r^{n-p}} \iff r=\sqrt[n-p]{\frac{\log_{u_p}(u_n)}{\log_{u_p}(u_p)}} \iff r=\sqrt[n-p]{\log_{u_p}(u_n)}\]

    Cela fonctionne donc $\forall u_p \in \mathbb{R_{+}^*}\backslash\lbrace{1}\rbrace$ et $\forall u_n \in \mathbb{R_{+}^*}$.
\end{proof}

\section*{Les variations des suites de puissances}

Soit une suite $u_n$, une suite de puissances de formule de récurrence $u_{n+1}=u_n^r$ avec $u_0=k$.
En considérant la méthode générale pour étudier les variations de suites, on trouve que:

    \[u_{n+1}-u_n \iff k^{r^{n+1}}-k^{r^n} \iff k^{r^n}(k^r-1)\]

Et l'étude des variations s'en suit. Voici un tableau résumant les variations possibles en fonction de $k^{r^n}$ et $k^{r^n}-1$. \\

\begin{tabularx}{12cm}{|c|p{1.5cm}|X|}
    \hline
    \hfil $r$ & \hfil $k$ & \hfil Variations de $u_n$ \\
    \hline
    $r>1$ & $k>1$ & $u_n$ est une suite strictement croissante. \\
    \hline
    $r>1$ & $k<0$ & $u_n$ est une suite strictement croissante pour $k$ étant pair et strictement décroissante pour $k$ étant impair. \\
    \hline
    $r>1$ & $k=0$ & $u_n$ est une suite constante ($u_n=0$). \\
    \hline
    $r>1$ & $k\in]0,1[$ & $u_n$ est une suite décroissante ($\lim_{n \to +\infty}u_n=0$). \\
    \hline
    $r>1$ & $k=1$ & $u_n$ est une suite constante ($u_n=1$). \\
    \hline
    $r<0$ & $k>1$ & $u_n$ n'est pas une suite monotone. \\
    \hline
    $r<0$ & $k<0$ & $u_n$ n'est pas une suite monotone ($u_n \in \mathbb{C}$ si $r \in ]-1,0[ \implies$ $u_n$ n'est pas continue sur $\mathbb{R}$). \\
    \hline
    $r<0$ & $k=0$ & $u_n$ n'est pas une suite monotone ni continue sur $\mathbb{R}$ ($u_n \in \mathbb{C}$, $u_n=0$ pour $n$ étant pair). \\
    \hline
    $r<0$ & $k\in]0,1[$ & $u_n$ n'est pas une suite monotone. \\
    \hline
    $r<0$ & $k=1$ & $u_n$ est une suite constante ($u_n=1$). \\
    \hline
    $r\in]0,1[$ & $k>1$ & $u_n$ est une suite décroissante ($\lim_{n \to +\infty}u_n=1$). \\
    \hline
    $r\in]0,1[$ & $k<0$ & $u_n$ n'est pas une suite monotone ni continue sur $\mathbb{R}$ ($u_n \in \mathbb{C}$). \\
    \hline
    $r\in]0,1[$ & $k=0$ & $u_n$ est une suite constante ($u_n=0$). \\
    \hline
    $r\in]0,1[$ & $k\in]0,1[$ & $u_n$ est une suite croissante ($\lim_{n \to +\infty}u_n=1$). \\
    \hline
    $r\in]0,1[$ & $k=1$ & $u_n$ est une suite constante ($u_n=1$). \\
    \hline
    $r=1$ & $k>1$ & $u_n$ est une suite constante ($u_n=k$). \\
    \hline
    $r=1$ & $k<0$ & $u_n$ est une suite constante ($u_n=k$). \\
    \hline
    $r=1$ & $k=0$ & $u_n$ est une suite constante ($u_n=0$). \\
    \hline
    $r=1$ & $k\in]0,1[$ & $u_n$ est une suite constante ($u_n=k$). \\
    \hline
    $r=1$ & $k=1$ & $u_n$ est une suite constante ($u_n=1$). \\
    \hline
\end{tabularx}

\end{document}